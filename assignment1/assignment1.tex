%----------------------------------------------------------------------------------------
%	PACKAGES AND OTHER DOCUMENT CONFIGURATIONS
%----------------------------------------------------------------------------------------

\documentclass{article}

\usepackage[utf8]{inputenc}
\usepackage{amssymb}
\usepackage{amsmath}
\usepackage{float}
\usepackage{latexsym}
\usepackage{subcaption}
\usepackage{gensymb}
\usepackage{caption}
\usepackage{fancyhdr} % Required for custom headers
\usepackage{lastpage} % Required to determine the last page for the footer
\usepackage{extramarks} % Required for headers and footers
\usepackage[usenames,dvipsnames]{color} % Required for custom colors
\usepackage{graphicx} % Required to insert images
\usepackage{listings} % Required for insertion of code
\usepackage{courier} % Required for the courier font
\usepackage{lipsum} % Used for inserting dummy 'Lorem ipsum' text into the template
\usepackage{tabularx} % For nice tables

%%%% NEW
\usepackage{color}

\definecolor{mygreen}{rgb}{0,0.6,0}
\definecolor{mygray}{rgb}{0.5,0.5,0.5}
\definecolor{mymauve}{rgb}{0.58,0,0.82}

\lstset{ %
  backgroundcolor=\color{white},   % choose the background color; you must add \usepackage{color} or \usepackage{xcolor}
  basicstyle=\footnotesize,        % the size of the fonts that are used for the code
  breakatwhitespace=false,         % sets if automatic breaks should only happen at whitespace
  breaklines=true,                 % sets automatic line breaking
  captionpos=b,                    % sets the caption-position to bottom
  commentstyle=\color{mygreen},    % comment style
  deletekeywords={...},            % if you want to delete keywords from the given language
  escapeinside={\%*}{*)},          % if you want to add LaTeX within your code
  extendedchars=true,              % lets you use non-ASCII characters; for 8-bits encodings only, does not work with UTF-8
  frame=single,	                   % adds a frame around the code
  keepspaces=true,                 % keeps spaces in text, useful for keeping indentation of code (possibly needs columns=flexible)
  keywordstyle=\color{blue},       % keyword style
  language=Octave,                 % the language of the code
  otherkeywords={*,...},           % if you want to add more keywords to the set
  numbers=left,                    % where to put the line-numbers; possible values are (none, left, right)
  numbersep=5pt,                   % how far the line-numbers are from the code
  numberstyle=\tiny\color{mygray}, % the style that is used for the line-numbers
  rulecolor=\color{black},         % if not set, the frame-color may be changed on line-breaks within not-black text (e.g. comments (green here))
  showspaces=false,                % show spaces everywhere adding particular underscores; it overrides 'showstringspaces'
  showstringspaces=false,          % underline spaces within strings only
  showtabs=false,                  % show tabs within strings adding particular underscores
  stepnumber=2,                    % the step between two line-numbers. If it's 1, each line will be numbered
  stringstyle=\color{mymauve},     % string literal style
  tabsize=2,	                   % sets default tabsize to 2 spaces
  title=\lstname                   % show the filename of files included with \lstinputlisting; also try caption instead of title
}


%%%% NEW

% Margins%
\topmargin=-0.45in
\evensidemargin=0in
\oddsidemargin=0in
\textwidth=6.5in
\textheight=9.0in
\headsep=0.25in

\linespread{1.1} % Line spacing

% Set up the header and footer
\pagestyle{fancy}
\lhead{AP 2016 - Exam} % Top left header
\chead{}
\lfoot{\lastxmark} % Bottom left footer
\cfoot{} % Bottom center footer
\rfoot{Page\ \thepage} % Bottom right footer
\renewcommand\headrulewidth{0.4pt} % Size of the header rule
\renewcommand\footrulewidth{0.4pt} % Size of the footer rule

\setlength\parindent{16pt} % Removes all indentation from paragraphs

\setcounter{secnumdepth}{0} % Removes default section numbers

%----------------------------------------------------------------------------------------
% TITLE PAGE
%----------------------------------------------------------------------------------------

\title{
\vspace{1in}
\textmd{\textbf{Machine Learning 2016 - Assignment 1}} \\
%\textmd{Assignment 2 - Resubmission} \\
\author{Christoffer Thrysøe - dfv107}
}

%----------------------------------------------------------------------------------------

\begin{document}
\maketitle
% \tableofcontents
\pagenumbering{arabic}
\section{1. Vectors and Matrices}
For this sub problem the following vectors and matrix were given :
$$
a = \begin{bmatrix}
1\\
2\\
2\end{bmatrix}, \textbf{ }
b = \begin{bmatrix}
3\\
2\\
1\end{bmatrix} and \textbf{ }
M = \begin{bmatrix}
1&0&0\\
0&4&0\\
0&0&2\end{bmatrix}
$$

\subsection{Question 1}
Compute $ a^Tb $:
$$
\begin{bmatrix}
1 & 2 & 2\end{bmatrix}
\begin{bmatrix}
3\\
2\\
1\end{bmatrix}
= (1 \times 3) + ( 2 \times 2) + (2 \times 1) = 9
$$
thus, $a^Tb=9$
\subsection{Question 2}
Compute the length of vector $\textbf{a}$:
$$ \|a\| = \sqrt{ 1^2 + 2^2 + 2^2} = 3 $$ 
thus, $\|a\| = 3$
\subsection{Question 3}
$$
\begin{bmatrix}
1\\
2\\
2\end{bmatrix}
\begin{bmatrix}
3 & 2 & 1\end{bmatrix}
=
\begin{bmatrix}
3 \times 1 & 2 \times 1 &1 \times 1 \\
3 \times 2 & 2 \times 2 & 1 \times 2 \\
3 \times 2 & 2 \times 2 & 1 \times 2 \end{bmatrix}
=
\begin{bmatrix}
3 & 2 & 1 \\
6 & 4 & 2 \\
6 & 4 & 2
\end{bmatrix}
$$
Therefore 
$a^Tb \neq ab^T$ so the answer is "No".
\subsection{Question 4}
$$
\begin{bmatrix}
3 & 2 & 1\end{bmatrix}
\begin{bmatrix}
1\\
2\\
2 \end{bmatrix}
= (3 \times 3) + ( 2 \times 2) + (1 \times 2) = 9
$$
Therefore $a^Tb = b^Ta$, and the answer is "Yes". 
\subsection{Question 5}
To calculate the inverse of matrix M, the matrix is brought to the echelon form, the same operations are applied to an Identity matrix I. The result of applying the operations to I will be the inverse of matrix M, if reduction is possible.
$$
\left[\begin{array}{rrr|rrr}
1&0&0&1&0&0\\
0&4&0&0&1&0\\
0&0&2&0&0&1
\end{array}\right]
$$
Row 2 is multiplied by $\frac{1}{4}$, row 3 is multiplied by $\frac{1}{2}$. The resulting matrix after the row operations are:
$$
\left[\begin{array}{rrr|rrr}
1&0&0&1&0&0\\
0&1&0&0& \frac{1}{4} &0\\
0&0&1&0&0&\frac{1}{2}
\end{array}\right]
$$
Therefore:
$$
\mathbf{M^{-1}} =
\begin{bmatrix}
1 & 0 & 0 \\
0 & \frac{1}{4} & 0 \\
0 & 0 & \frac{1}{2}
\end{bmatrix}
$$
\subsection{Question 6}
Compute $ \textbf{Ma}$:
$$
\begin{bmatrix}
1 & 0 & 0 \\
0 & 4 & 0 \\
0 & 0 & 2
\end{bmatrix}
\begin{bmatrix}
1 \\
2 \\
2
\end{bmatrix}
=
\begin{bmatrix}
(1 \times 1) + (1 \times 0) + (1 \times 0) \\
(2 \times 0) + (2 \times 4) + (2 \times 0)  \\
2 \times 0)  + (2\times 0) + (2 \times 2) 
\end{bmatrix}
=
\begin{bmatrix}
1 \\
8 \\
4
\end{bmatrix}
$$
Therefore:
$$
\textbf{Ma} =
\begin{bmatrix}
1 \\
8 \\
4
\end{bmatrix}
$$
\subsection{Question 7}
$$
\mathbf{A} =  
\begin{bmatrix}
3 & 2 & 1 \\
6 & 4 & 2 \\
6 & 4 & 2
\end{bmatrix}
\textbf{, }
\mathbf{A^T}
\begin{bmatrix}
3 & 6 & 6 \\
2 & 4 & 4 \\
1 & 2 & 2
\end{bmatrix}
$$
\textbf{A} is not symmetric as $\mathbf{A} \neq \mathbf{A^T}$.
\subsection{Question 8}
The matrix a canbe reduced to echelon form, by subtracting two times row 1 from row 2 and one time row 2 from row 3.
$$
\begin{bmatrix}
3 & 2 & 1 \\
6 & 4 & 2 \\
6 & 4 & 2
\end{bmatrix}
\rightarrow
\begin{bmatrix}
3 & 2 & 1 \\
0 & 0 & 0 \\
0 & 0 & 0
\end{bmatrix}
$$
The number of linearly independent rows in the reduced matrix is 1, therefore $\mathbf{Rank}(\mathbf{A}) = 1$
\subsection{Question 9}
In order for a square matrix to be invertible, it must have full rank, that is for a matrix of size $m \times m$ the rank should be m.
\section{2. Derivatives}
\subsection{Question 1}
\begin{align*}
f(x) &= \dfrac{1}{e^{-x} +1} \\ \\
\frac{\partial}{\partial x} f(x) &=  \frac{\partial}{\partial x}
  \left[\dfrac{1}{e^{-x} +1}\right] \\ \\
&= - \dfrac{ \frac{\partial}{\partial x} \left[e^{-x}+1 \right] }
           {(e^{-x}+1)^2} \textsl{  \hspace{1cm}  (Reciprocal rule)}\\ \\
&= - \dfrac{\frac{\partial}{\partial x} \left[e^{-x} \right] + \frac{\partial}{\partial x} \left[1 \right]}
           {(e^{-x}+1)^2}\\ \\
&= - \dfrac{ e^{-x} \frac{\partial}{\partial x} \left[-x \right] +0}
           {(e^{-x}+1)^2} \textsl{  \hspace{0,6cm}  (Exponential function rule)} \\ \\
&= - \dfrac{ e^{-x} -\frac{\partial}{\partial x} \left[x \right]}
           {(e^{-x}+1)^2} \\ \\
&= \dfrac{ e^{-x} }
           {(e^{-x}+1)^2}
\end{align*}
\subsection{Question 2}
\begin{align*}
f(w,x) &= 2 (wx+5)^2 \\ \\
\frac{\partial}{\partial w} f(w,x) &= \frac{\partial}{\partial w} \left[2(wx+5)^2\right] \\ \\
&= 2 \frac{\partial}{\partial w} \left[(wx+5)^2\right] \\ \\
&= 2 \times 2 (wx+5) \frac{\partial}{\partial w} \left[wx+5\right] \textsl{  \hspace{1cm}  (Power rule)} \\ \\
&= 4 (x \frac{\partial}{\partial w} \left[ w \right] + \frac{\partial}{\partial w} \left[ 5 \right]) (wx+5)\\ \\
&= 4(1x+0)(wx+5) \\ \\
&= 4x(wx+5) \\ \\
\end{align*}
\section{3. Probability Theory: Sample Space}
\subsection{Question 1}
The sample space $\Omega$ is the set of possible outcomes. \\
There are three different colours, 9 balls. 2  balls are picked from the urn uniformly at random. In the urn there is 5 red balls, 3 orange balls and 1 blue ball. Each colour is denoted by its starting letter in capital. The size of the sample space is $|\Omega| = \texttt{colours}^{\texttt{picks}} = 3^2 = 9$, the possible outcomes are:
$$ \Omega = \left\lbrace \texttt{RR, RO, RB , OO, OR, OB, BB, BR, BO} \right\rbrace $$
% NOTE: BB not a possible outcome should not be in sample space.
\subsection{Question 2}
\begin{table}[H]
  \centering
  \caption{Caption for the table.}
  \label{tab:table1}
  \begin{tabular}{c|c|c||r}
    \texttt{event} & $Pr[\texttt{"First draw"}]$ & $Pr[\texttt{"Second draw"}]$ & $Pr[\texttt{event}]$ \\
    \hline
    RR & $5/9$ & $4/8$ &  $ 5/9 \times 4/8 = 5/18$\\
    RO & $5/9$ & $3/8$ &  $ 5/9 \times 3/8 = 5/24$\\
    RB & $5/9$ & $1/8$ &  $ 5/9 \times 1/8 = 5/72$\\
    OO & $3/9$ & $2/8$ &  $ 3/9 \times 2/8 = 1/12$\\
    OR & $3/9$ & $5/8$ &  $ 3/9 \times 5/8 = 5/24$\\
    OB & $3/9$ & $1/8$ &  $ 3/9 \times 1/8 = 1/24$\\
    BB & $1/9$ & $0/8$ &  $ 1/9 \times 0/8 = 0$\\
    BR & $1/9$ & $5/8$ &  $ 1/9 \times 5/8 = 5/72$\\
    BO & $1/9$ & $3/8$ &  $ 1/9 \times 3/8 = 1/24$\\
  \end{tabular}
\end{table}
\subsection{Question 3}
As two balls are picked, and the urn contains 3 orange balls, the the possible values for $X$ is:
$$ X \in \left\lbrace 0,1,2 \right\rbrace $$
\subsection{Question 4}
Compute $Pr[X = 0]$, that is the probability that the two balls picked are either "red" or "blue". I will calculate the probability of $X = 0$, by calculating the probability of not picking an orange ball in the first and second draw, that is the complement of those events happening:
$$ Pr[\texttt{"not orange"}] = 1-(3/9) \times 1-(3/8) = 5/12 $$
therefore:
$$ Pr[X=0] = 5/12 $$
\subsection{Question 5}
To compute the expected value of $X$, i will use the general formula:
\begin{equation}
\mathbf{E}[X] = \sum\limits_{x \in \mathbf{X}} x  Pr[\mathbf{X} = x]
\label{expectation}
\end{equation}
We saw from question 3 that the possible outcomes for $X$ is 0,1 or 2, to get the probability of each value, each sample space satisfying one of the values are summed for the specific value.
\begin{align*}
Pr[X=0] &= \dfrac{5}{18} + \dfrac{5}{72} + \dfrac{5}{72} = \dfrac{5}{12} \\
Pr[X=1] &= \dfrac{5}{24} + \dfrac{5}{24} + \dfrac{1}{24} + \dfrac{1}{24} = \dfrac{1}{2} \\
Pr[X=2] &= \dfrac{1}{12} \\
\end{align*}
The numbers can now be put into \eqref{expectation}:
$$
\mathbf{E}[X] = (0 \times \dfrac{5}{12}) + (1 \times \dfrac{1}{2}) + (2 \times \dfrac{1}{12}) = \dfrac{2}{3}
$$
Thus the expected number of orange balls picked is $ \dfrac{2}{3}$
% NOTE: seems wrong that it is less than one
\section{4. Probability Theory: Properties of Expectation }
\subsection{Question 1}
Proof of: $ \mathbf{E}[X+Y]= \mathbf{E}[X] + \mathbf{E}[Y]$
\begin{align}
\mathbf{E}[X,Y] &= \sum\limits_{x,y}(x+y)Pr(X=x,Y=y) \\
&= \sum\limits_{x,y} x Pr(X=x,Y=y) + \sum\limits_{x,y} y Pr(X=x,Y=y) \\
&= \sum\limits_{x} x \sum\limits_{y} Pr(X=x,Y=y) + \sum\limits_{y} y \sum\limits_{x}  Pr(X=x,Y=y) \\
&= \sum\limits_{x} x Pr(X=x) + \sum\limits_{y} y Pr(Y=y) \texttt{   (By rule of total probability)}\\
&= \mathbf{E}[X] + \mathbf{E}[Y]
\end{align}
Step 3 to 4 uses the rule of total probability : $ \sum\limits_{x} Pr(X=x,Y=y) = Pr(X=x) $
\subsection{Question 2}
Proof of: $ \mathbf{E}[XY]= \mathbf{E}[X] \mathbf{E}[Y]$
\begin{align}
\mathbf{E}[XY] &= \sum\limits_{x,y} Pr(X=x,Y=y) \\
&= \sum\limits_{x}\sum\limits_{y} xy Pr(X=x,Y=y) \\
&= \sum\limits_{x}\sum\limits_{y} xy Pr(X=x)Pr(Y=y) \texttt{  (Independent)} \\
&= \sum\limits_{x}x Pr(X=x)\sum\limits_{y} y Pr (Y=y) \\
&= \mathbf{E}[X] \mathbf{E}[Y] \\
\end{align}
where step 8 follows from X and Y being independent random variables.
\subsection{Question 3}
\subsection{Question 4}
Proof of $ \mathbf{E}[\mathbf{E}[X]] = \mathbf{E}[X] $
\subsection{Question 5}
Proof of $\mathbf{E}\left[(X-\mathbf{E}[X])^2 \right] = 
\mathbf{E}[X^2]-\left( \mathbf{E}[X] \right) ^2:
$
\begin{align*}
\mathbf{E}\left[(X-\mathbf{E}[X])^2 \right] &=
\mathbf{E}\left[X^2-2\mathbf{E}[X]X+  \left(\mathbf{E}[X]\right)^2 \right]\\
&=
\mathbf{E}[X^2]-2\mathbf{E}[X]\mathbf{E}[X]+\left(\mathbf{E}[X]\right)^2 \\
&=
\mathbf{E}[X^2]-2 \left(\mathbf{E}[X]\right)^2 +\left(\mathbf{E}[X]\right)^2 \\
&=
\mathbf{E}[X^2]-\left(\mathbf{E}[X]\right)^2\\
\end{align*}
Thus proving that $\mathbf{E}\left[(X-\mathbf{E}[X])^2 \right] = 
\mathbf{E}[X^2]-\left( \mathbf{E}[X] \right) ^2$.


\section{5. Probability Theory: Complements of Events}
\subsection{Question 1}
We denote the complement of an event A by $\overline{A}$. We define the complement as $\overline{A} = \Omega \diagup A$, thus noting that $A$ and its complement are mutually exclusive, meaning we can write $Pr[A\cup B] = Pr[A] + Pr[B]$. \\
The following is a proof of $Pr[A] = 1 - Pr[\overline{A}]$:
\begin{align*}
1 &= Pr[S] = Pr[A \cup \overline{A}] = Pr[A]+ Pr[\overline{A}] \\
1 &= Pr[A]+ Pr[\overline{A}] \\
Pr[A] &= 1 - Pr[\overline{A}]
\end{align*}
Thus concluding that $Pr[A] = 1 - Pr[\overline{A}]$.
\subsection{Question 2}
If a coin is flipped 10 times, what is the probability of at least one flip being tail. To solve this problem we first denote the aformentioned event as $A = \texttt{"at least one tail"}$, to find the probability of event $A$ i will calculate it's complement which in the above case corresponds to the event $\overline{A} = \texttt{"all heads"}$, the probability of this event can be calculated as followed:
\begin{align}
Pr[A] &= 1-Pr[\overline{A}] \\
&= 1-Pr[\texttt{"all heads"}] \\
&= 1 - \left( Pr[H_1] \times Pr[H_2] \times ... \times Pr[H_{10}] \right) \\
&= 1 - \left( \dfrac{1}{2} \right)^{10} \approx 0.999
\end{align}
Thus the probability of getting at least one tail is very large. \\
To calculate the probability of observing at least two tails i will calculate the binomial distribution, which takes parameters n and k, where k is the number of successes in the sequence of n independent experiments. The binomial distribution is calculated as followed:
$$
Pr[n,k] = \binom{n}{k}p^k(1-p)^{n-k}
$$
For the given example the binomial distribution can be calculated as:
$$
Pr[10,2] = \binom{10}{2} \left(\dfrac{1}{2}^2\right)\left(1-\dfrac{1}{2}\right)^8
\approx 0.956
$$
% NOTE: Perhaps should be added with for one as it is at least, no maybe not 
\section{6. Probability Theory: Coin Flips}
A coin is flipped 10 times the following sub assignments about finding probabilities of different events.
\subsection{Question 1}
As the coin was flipped 10 times, the number of heads must be equal to the number of tails iff. there are 5 heads and 5 tails. Again i will use the binomial distribution to calculate the probability of the event:
$$
Pr[10,5] = \binom{10}{5} \left(\dfrac{1}{2}^5\right)\left(1-\dfrac{1}{2}\right)^5
\approx 0.246
$$
\subsection{Question 2}
The probability of getting more heads than tails will be calculated using binomial distruibtion. The probability must be added for the events of getting 6,7,8,9 and 10 heads, as any of these events will satisfy the question. This probability can be expressed as followed:
$$
Pr[\texttt{"more heads than tails"}] = \dfrac{\binom{10}{6} + \binom{10}{7} + \binom{10}{8} + \binom{10}{9} + \binom{10}{10}}{2^{10}} \approx 0.376
$$
\subsection{Question 3}
The \textit{ith} flip and the (11-i)\textit{th} flip are the same for i = 1,2,3,4,5. Flipping the coin for i = 1,2,3,4,5 generates $2^5$ possible outcomes. Thus yielding the probability:
$$
\dfrac{2^5}{2^{10}} = \dfrac{1}{32}
$$

\end{document}





